\section{Apéndices}
	\subsection{Apéndice I: generación y análisis de datos}

	Para poder analizar las complejidades de los algoritmos propuestos, se utilizaron las siguientes herramientas:

	\begin{itemize}
		\item Un generador de grafos aleatorios con \texttt{std::random} (parte de la biblioteca estándar de C++11).

		Para poder probar las distintas implementaciones, creamos un generador de grafos aleatorios. El mismo consiste básicamente en un generador de grafos completos y una función de mezclado aleatorio, basado en los generadores de distribuciones uniformes provistas por la \texttt{std} de C++.

		Cada uno de los grafos generados se utilizó multiples veces para testear varios aspectos de un mismo algoritmo, al igual que para realizar pruebas comparativas entre las distintas implementaciones.

		\item Un generador de grafos para casos patológicos

		Para analizar el comportamiento de nuestros algoritmos frente a ciertas instancias particulares, creamos un generador de grafos que no es aleatorio, sino que dados ciertos parámetros genera determinísticamente grafos poco favorables de las heurísticas.

		Dados $k$ y $q$, $k$ es el mayor grado entre los nodos del grafo, existen $q$ componentes donde hay un nodo de grado $k$ donde sus adyacentes son de grado 1, y además una componente tal que es una clique de $\frac{k}{2}$ nodos y cada uno de ellos tienen $\frac{k}{2}$ nodos adyacentes de grado 1(Como los descritos en la sección de casos patológicos).

		Estos grafos también se encuentran descritos en el análisis de las heurísticas y sus casos patológicos.

		\item Mediciones de tiempo con \texttt{std::chrono} (parte de la biblioteca estándar de C++11).

		Cada algoritmo fue probado 100 veces consecutivas con cada entrada, conservando solo el valor de tiempo menor para reducir el ruido por procesos ajenos al problema.

		La unidad de medición preferida fue nanosegundos (\texttt{std::chrono::nanoseconds}, $seg \times 10^{-9}$).

		\item Graficado con \texttt{matplotlib.pyplot}, \texttt{pandas} y \texttt{seaborn} (con Python y Jupyter Notebook)

		Se utilizaron los DataFrames de Pandas para el manejo de datos (guardados en \texttt{.csv}) y las funciones de regresión de Seaborn para el graficado, en conjunto con matplotlib.

	\end{itemize}

	Algunos gráficos incluyen coeficientes de correlación o ``R'' de Pearson. El mismo denota la correlación lineal entre 2 variables aleatorias. Nostros lo aplicamos entre los datos obtenidos (en particular, el tiempo de ejecución) y la curva graficada que se aproxima a esos datos. De esta manera, podemos determinar si nuestra aproximación es correcto: un R cercano o igual a 1 indica que hay una fuerte correlación positiva entre los valores graficados y la curva, es decir, la curva aproxima correctamente al gráfico. Junto al coeficiente se incluye un p-valor para la hipotesis nula que los valores pueden haber sido generados sin correlación real; un p-valor elevado invalidaría una posible correlación positiva o negativa, lo cual significaría que nuestra aproximación es erronea.

	\subsection{Apéndice II: herramientas de compilación y testing}
	Durante el desarrollo se utilizaron las siguientes herramientas:

	\begin{itemize}
		\item CMake

		Se decidió utilizar CMake para la compilación por su simplicidad y compatibilidad con otras herramientras. Junto con el código se provee el archivo \texttt{CMakeLists.txt} para compilar el mismo.

		\item Google Test

		Para generar tests unitarios con datos reutilizables se usó Google Test. Dichos archivos eran importados por otro \texttt{CMakeLists.txt} y no están incluídos en la presente entrega del trabajo práctico.

		\item Namespace \texttt{Utils}

		Dentro de \texttt{Utils.h} se definieron algunas funciones ajenas a los algoritmos en sí e independientes de implementación. Entre ellas se encuentran fuciones de parseo de input, así como una función de logging que fue utilizada al programar para detectar errores y ver otros detalles del proceso.

		La función \texttt{log} sigue estando incluída en los algoritmos, pero su funcionalidad se encuentra apagada por un \texttt{\#define} y no debería generar ningún costo adicional (ya que usa printf por detrás y no computa el output salvo que sea necesario).
	\end{itemize}