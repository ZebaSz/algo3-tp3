\section{Descripcion de situaciones reales}
El problema que vamos a analizar durante todo el informe es el de \textit{clique de máxima frontera}(\textit{CMF}) en un grafo, que consiste en hallar la clique tal que su frontera sea de cardinalidad máxima. Definimos la frontera de una clique K como el conjunto de aristas que tienen un extremo en K y otro por fuera de la clique.

Distintas situaciones problemáticas de la vida real podrían ser representadas por CMF, tales como:

\begin{itemize}

\item MarencoGames desarrolla un juego online y quiere que haya la menor cantidad de gente haciendo trampa. En este juego los jugadores juegan a través de la red, pero no todos están comunicados entre sí. Los desarrolladores quieren hacer un sistema de protección de trampa, por lo que deciden convertir varios jugadores en moderadores, los cuales van a reportar cuales de los participantes a los que están conectados están haciendo trampa. Los moderadores tienen que estar todos conectados entre sí para reportarse las trampas, y lo que se desea es cubrir la mayor cantidad de jugadores posibles.  Cada jugador puede ser representado con un nodo, y la conexión entre dos jugadores a través de aristas. Con esta representación, tendríamos que una clique en el grafo podrá ser un conjunto de moderadores conectados entre sí, y la frontera será la cantidad de jugadores que abarca el sistema de protección.

%TODO poner un problema mas

\end{itemize}


