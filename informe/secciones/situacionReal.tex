\section{Descripcion de situaciones reales}
El problema que vamos a analizar durante todo el informe es el de \textit{clique de máxima frontera}(\textit{CMF}) en un grafo, que consiste en hallar la clique tal que su frontera sea de cardinalidad máxima. Definimos la frontera de una clique K como el conjunto de aristas que tienen un extremo en K y otro por fuera de la clique.

Distintas situaciones problemáticas de la vida real podrían ser representadas por CMF, tales como:

\begin{itemize}

\item MarencoGames desarrolla un juego online y quiere que haya la menor cantidad de gente haciendo trampa. En este juego los jugadores juegan a través de la red, pero no todos están comunicados entre sí. Los desarrolladores quieren hacer un sistema de protección de trampa, por lo que deciden convertir varios jugadores en moderadores, los cuales van a reportar cuales de los participantes a los que están conectados están haciendo trampa. Los moderadores tienen que estar todos conectados entre sí para reportarse las trampas, y lo que se desea es cubrir la mayor cantidad de jugadores posibles.  Cada jugador puede ser representado con un nodo, y la conexión entre dos jugadores a través de aristas. Con esta representación, tendríamos que una clique en el grafo podrá ser un conjunto de moderadores conectados entre sí, y la frontera será la cantidad de jugadores que abarca el sistema de protección.

\item En un proyecto en conjunto llevado a cabo por alumnos de Biología y Computación, se quiere desarrollar un robot que tenga un comportamiento similar al de las arañas. Para esto, los biólogos realizaron una investigación y llegaron a la conclusión de que los arácnidos, a la hora de situarse en sus telarañas, buscan cumplir dos requisitos: mantenerse seguras, y alcanzar sus presas con total rapidez. De este modo, a los computólogos se les ocurrió imaginar la telaraña como un grafo, donde cada hilo será una arista y cada punta de hilo un vértice. De este modo, como son los hilos los que protegen a la araña, un lugar seguro será aquel donde todas las puntas tengan un hilo entre sí, y una punta alcanzable será todo vértice que se encuentre relacionado a un hilo que llegue hacia donde esta ella. Nuestra araña, de manera instintiva, no permitirá que ninguna presa se acerce adonde esta situada, por lo que solo consideraremos los sitios alcanzables por afuera de su lugar seguro. Como los alumnos concluyeron que sería más divertido tener una araña violenta, optaron por configurarla de manera que priorice el alcance a las presas antes que la seguridad, de modo que busque la posición segura donde más puntas alcanzables haya.

%TODO poner varios problemas mas

\end{itemize}


